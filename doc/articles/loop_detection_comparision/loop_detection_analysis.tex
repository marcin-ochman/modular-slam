\documentclass[conference]{IEEEtran}
\IEEEoverridecommandlockouts
% The preceding line is only needed to identify funding in the first footnote. If that is unneeded, please comment it out.
\usepackage{cite}
\usepackage{amsmath,amssymb,amsfonts}
\usepackage{algorithmic}
\usepackage{graphicx}
\usepackage{textcomp}
\usepackage{xcolor}
\usepackage[utf8]{inputenc}
\usepackage{hyperref}

\def\BibTeX{{\rm B\kern-.05em{\sc i\kern-.025em b}\kern-.08em
    T\kern-.1667em\lower.7ex\hbox{E}\kern-.125emX}}
\begin{document}

\title{Evaluation of loop detection algorithms in outdoor environment.}

\author{\IEEEauthorblockN{Magda Skoczeń}
\IEEEauthorblockA{\textit{Faculty of Electronics} \\
\textit{Wroclaw University of Science and Technology}\\
Wroclaw, Poland \\
\href{https://orcid.org/0000-0003-4863-7935}{https://orcid.org/0000-0003-4863-7935}}
\and
\IEEEauthorblockN{ Marcin Ochman}
\IEEEauthorblockA{\textit{Faculty of Electronics} \\
\textit{Wroclaw University of Science and Technology}\\
Wroclaw, Poland \\
 \href{https://orcid.org/0000-0002-8075-0033}{https://orcid.org/0000-0002-8075-0033}}
}

\maketitle

\begin{abstract}
  One of the most crucial SLAM components is loop closure detection. Its main task is detecting
  already visited places and providing additional constraints for localization
  module to reach better accuracy. Visual SLAM reaches more and more research interest. As a
  result, loop closure detection became a computer vision
  problem. Representing seen scenes and determining whether place has been
  visited or not is challenging.

  This paper aims to examine numerous feature points descriptors in terms of loop
  detection algorithms performance. A number of classification metrics were given as well as
  execution time of descriptors calculation. Furthermore, various approaches of
  loop closures detection have been tested. The final contribution of this
  work is a dataset containing image sequences of outdoor, considered
  as a more difficult environment.

\end{abstract}

\begin{IEEEkeywords}
  Visual SLAM, Loop Closure, Image Descriptors, Feature Detection
\end{IEEEkeywords}

\section{Introduction}
\section{Related Work}
\section{Experimental setup}
\section{Experimental results}
\section{Discussion}
\section{Conclusion}



\end{document}
